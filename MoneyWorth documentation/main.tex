\documentclass[12pt]{article}
\usepackage[utf8]{inputenc}
\usepackage{lmodern} % Ensures all font shapes are available
\usepackage{graphicx}
\usepackage{hyperref}
\usepackage{minted}
\usepackage{float}
\usepackage{geometry}
\geometry{margin=1in}


\title{Moneyworth Web App Project Report}
\author{Kazibwe David Nelson (2400724054) \and Sseruwagi Don Marvin (2400724775) 
\and Mila Samantha Likiya (2400724201) \and Mayanja Joel Stephen (2400724184)}
\date{May 15, 2025}

\begin{document}
\maketitle

\section{Problem Statement}
Sales managers struggle to create, organize, and follow sales records through traditional or ad-hoc products,
which creates unstructured data, reduced efficiency, and lost opportunities. A contemporary, Web-based system, 
which can make these operations easy within a Web 2.0 environment, is required. Present systems are generally 
devoid of integrated authentication, inherent selling reporting, and interactive plotting, and follow double
 verification schemes.\textbf{Moneyworth} is presented for filling the gaps.It has the following pages:

\subsection{Home}
The Home page features a welcoming jumbotron section with summary cards. Each card includes a call-to-action 
button linking to either the ``Create Sale'' or ``Browse Sales'' page. This design helps users quickly access
 the main functions of the app.
\begin{figure}[H]
    \centering
    \includegraphics[width=0.5\textwidth]{home.png}
    \caption{The Home page of the application, featuring navigation and summary cards.}
\end{figure}

\subsection{Dashboard}
The Dashboard page displays summary cards that provide an overview of sales performance:
\begin{itemize}
    \item \textbf{Total sales:} The total number of sales recorded in the system.
    \item \textbf{Active sales:} The number of sales that are fully paid and currently in progress.
    \item \textbf{Pending sales:} The number of sales with partial payments.
    \item \textbf{Completed sales:} The number of sales that are fully paid and completed.
\end{itemize}
The Dashboard also includes a table of recent sales and call-to-action (CTA) buttons that link to the 
``Create Sale'' and ``Browse Sales'' pages for quick access.
\begin{figure}[H]
    \centering
    \includegraphics[width=0.5\textwidth]{dashboard.png}
    \caption{The Dashboard page displaying user statistics and performance charts.}
\end{figure}

\subsection{Browse Sales}
The Browse Sales page presents a table of all sales records, each identified by a unique ID. 
It also provides a search bar for filtering results and a CTA button that directs the user to the Create Sale page. This page allows users to quickly find and review existing sales entries.
\begin{figure}[H]
    \centering
    \includegraphics[width=0.5\textwidth]{browse_sales.png}
    \caption{The Browse Sales page showing a table of all sales records.}
\end{figure}

\subsection{Create Sale}
The Create Sale page contains a form with fields for entering details of a new sale, 
such as the product name, quantity, and price. This page enables users to add new sales 
records to the system.
\begin{figure}[H]
    \centering
    \includegraphics[width=0.5\textwidth]{create_sales.png}
    \caption{The Create Sale page with a form for adding a new sales record.}
\end{figure}

\subsection{About}
The About page provides a brief introduction to the Moneyworth sales app. 
It highlights the team behind the development of the application and the 
vision and mission of the project.
\begin{figure}[H]
    \centering
    \includegraphics[width=0.5\textwidth]{about_us.png}
    \caption{The About page of the application, describing its purpose and development team.}
\end{figure}

\subsection{Contact}
The Contact page provides users with a form to reach out to the development team or support staff. 
It includes fields for the user's name, email, and message, facilitating communication between users
 and developers.
\begin{figure}[H]
    \centering
    \includegraphics[width=0.5\textwidth]{contact_us.png}
    \caption{The Contact page offering a form to contact the development team.}
\end{figure}

\section{HTML and Django Template Code Implementation}
This section presents the HTML and Django template code used to implement the project's layout.

\subsection{Base Template Explanation}
The base template, \texttt{base.html}, defines the overall structure of the site. 
It includes a navigation bar, a hero (jumbotron) section on the Home page, and a footer. 
Content blocks are defined using Django's template tags, allowing child templates to override 
specific sections. For example:

\begin{minted}[fontsize=\small]{html}

<!DOCTYPE html>
<html lang="en">
<head>
    <meta charset="UTF-8">
    <title>Django Sales App</title>
    <link rel="stylesheet" href="">
    <link rel="stylesheet" href="">
    
</head>
<body>
    <nav>...navbar code...</nav>
    
    <footer>...footer code...</footer>
    <script src=""></script>
</body>
</html>
\end{minted}

\subsection{Child Template Example}
A child template (for instance, \texttt{home.html}) extends the base template and fills in the 
designated content blocks:

\begin{minted}[fontsize=\small]{html}


Home - Django Sales App


<div class="jumbotron">
    <h1>Welcome to the Django Sales App!</h1>
    <p>Browse and manage sales listings easily.</p>
</div>

\end{minted}

\subsection{Use of Template Tags}
Key Django template tags used in this project include:
\begin{itemize}
    \item \verb||: Allows a template to inherit from a base template.
    \item \verb||: Defines sections of content that child templates can override.
    \item \verb||: Resolves the URL for static files (CSS, JavaScript, images).
\end{itemize}
These tags help organize the templates, promote code reuse, and simplify static file management.

\subsection{URL and View Structure}
The \texttt{urls.py} and \texttt{views.py} files connect URLs to view functions and templates. For example:

\begin{minted}[fontsize=\small]{python}
# core/urls.py
from django.urls import path
from . import views

urlpatterns = [
    path('', views.home, name='home'),
    path('dashboard/', views.dashboard, name='dashboard'),
    path('sales/', views.browse_sales, name='browse_sales'),
    path('sales/create/', views.create_sale, name='create_sale'),
    path('about/', views.about, name='about'),
    path('contact/', views.contact, name='contact'),
]
\end{minted}

\begin{minted}[fontsize=\small]{python}
# core/views.py
from django.shortcuts import render

def home(request):
    return render(request, 'core/home.html')

# ...other view functions...
\end{minted}

\section{Benefits of Django for Responsive Web Apps}
Django is ideal for responsive web apps due to its clear separation of logic and presentation 
(MTV architecture), easy integration with Bootstrap for mobile-friendly design, and built-in 
support for security, scalability, and fast development.

\section{Linking Static CSS and JavaScript in Django}
To include static files (CSS and JavaScript) in Django:
\begin{enumerate}
    \item Set \texttt{STATIC\_URL} in \texttt{settings.py} (e.g., \texttt{STATIC\_URL = '/static/'}).
    \item Add \verb|| at the top of each template.
    \item Reference static files using \verb|| 
    (for example, \verb||).
\end{enumerate}
For example:

\begin{minted}[fontsize=\small]{html}

<link rel="stylesheet" href="">
\end{minted}

\section{Components of MTV Architecture in Django}
Django follows the Model-Template-View (MTV) architecture:
\begin{itemize}
    \item \textbf{Model:} Defines the data structure. For example, the \texttt{Sales} model in 
    \texttt{core/models.py} manages the sales data.
    \item \textbf{View:} Handles requests and returns responses. In this project, functions in 
    \texttt{core/views.py} render templates for each page based on the request.
    \item \textbf{Template:} Renders the HTML presentation. Template files 
    (e.g., in \texttt{templates/core/}) define what the user sees in the browser.
\end{itemize}

\end{document}
