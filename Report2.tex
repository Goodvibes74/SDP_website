% Compile this document with XeLaTeX to use the Poppins font
\documentclass[12pt]{article}
\usepackage{fontspec}                   % For custom fonts
\setmainfont{Poppins}                   % Set Poppins as main font
\usepackage{geometry}                   % Page margins
\geometry{margin=1in}
\usepackage{hyperref}
\usepackage{listings}                   % For code listings
\lstset{
    basicstyle=\ttfamily\small,
    breaklines=true
}

\title{Software Development Project - Coursework Report}
\author{Your Name \\ Makerere University}
\date{\today}

\begin{document}
\maketitle
\tableofcontents
\newpage

\section{Introduction}
This report details the improvements made to the Django-based sales application, focusing on templates and MVC structure. We follow the academic structure guidelines, using the Poppins font for a modern look. All code snippets (HTML templates, URLs, and views) are included in the body.

\section{Implementation}

\subsection{Base Template (\texttt{base.html})}
The base template provides the site-wide layout (navigation bar, footer, etc.) and uses Bootstrap for responsiveness. For example, we load static assets with \texttt{\ and define a content block for child templates. The relevant code is shown below:
\begin{lstlisting}[language=HTML]
 Base template with navbar, container, and footer 

<!DOCTYPE html>
<html>
<head>
    <title>SaleApp</title>
    <link rel="stylesheet" href="">
    <link rel="stylesheet" href="">
</head>
<body>
     Navbar with brand and links 
    <nav class="navbar navbar-expand-lg navbar-light bg-light">
      ...
    </nav>
     Main content block 
    <div class="container my-4">
        
    </div>
     Footer with social icons 
    <footer class="footer bg-light text-center py-3">
        &copy; 2025 SaleApp
    </footer>
    <script src=""></script>
</body>
</html>
\end{lstlisting}

\subsection{Child Templates}
Child pages extend the base template. For instance, **home.html** adds a hero section using a Bootstrap jumbotron:
\begin{lstlisting}[language=HTML]


<div class="jumbotron text-center bg-light">
    <h1 class="display-4">Welcome to SaleApp!</h1>
    <p class="lead">Find great deals or sell your items easily.</p>
    <a class="btn btn-primary btn-lg" href="">
        Browse Sales
    </a>
</div>

\end{lstlisting}
The **create_sale.html** and **browse_sale.html** templates include forms and loops to match the `Sale` model fields (`title`, `description`, `price`). Django comments annotate each part. 

\subsection{URL and View Configuration}
Example Django URL patterns and view functions are shown to demonstrate the MTV architecture. In \texttt{urls.py}:
\begin{lstlisting}[language=Python]
from django.urls import path
from . import views

urlpatterns = [
    path('', views.home, name='home'),
    path('create/', views.create_sale, name='create_sale'),
    path('browse/', views.browse_sales, name='browse_sales'),
    path('dashboard/', views.dashboard, name='dashboard'),
]
\end{lstlisting}
And in \texttt{views.py}:
\begin{lstlisting}[language=Python]
from django.shortcuts import render, redirect
from .models import Sale

def home(request):
    return render(request, 'core/home.html')

def create_sale(request):
    if request.method == 'POST':
        # Create new Sale from POST data
        title = request.POST.get('title')
        desc = request.POST.get('description')
        price = request.POST.get('price')
        Sale.objects.create(title=title, description=desc, price=price)
        return redirect('browse_sales')
    return render(request, 'core/create_sale.html')

def browse_sales(request):
    sales = Sale.objects.all()
    return render(request, 'core/browse_sale.html', {'sales': sales})

def dashboard(request):
    total = Sale.objects.count()
    return render(request, 'core/dashboard.html', {'total_sales': total})
\end{lstlisting}

\section{Conclusion}
The updated templates now align with the coursework instructions: they use Bootstrap for a clean, responsive design, include all required form fields and content, and are well-commented. The LaTeX report template is fully structured (title page, TOC, sections) and uses Poppins font via `fontspec` (which enables OpenType fonts in XeLaTeX&#8203;:contentReference[oaicite:5]{index=5}). Code snippets are formatted with the `listings` package for readability. 

\begin{thebibliography}{9}
\bibitem{django-static} Django documentation, *How to manage static files*, referring to using `` and `STATIC_URL` (Django 5.2)&#8203;:contentReference[oaicite:6]{index=6}.
\bibitem{bootstrap-grid} Bootstrap Documentation, *Grid system* (Bootstrap 4/5) – describes containers, rows, and columns for responsive layout&#8203;:contentReference[oaicite:7]{index=7}.
\bibitem{fontspec} Will Robertson, *fontspec – Advanced font selection in XeLaTeX and LuaLaTeX* (CTAN), describing how `fontspec` enables OpenType fonts like Poppins&#8203;:contentReference[oaicite:8]{index=8}.
\end{thebibliography}

\end{document}
