% !TeX root = Report.tex
\documentclass[12pt]{article}
\usepackage[margin=1in]{geometry}
\usepackage{hyperref}
\usepackage{amsmath}
\title{Moneyworth Web App Project Report}
\author{Group Members: Name1, Name2, Name3, Name4}
\date{\today}

\begin{document}
\maketitle

\section{Introduction}
Moneyworth is a web application designed to help sale managers 
create and track their sales listings.
The main features include user authentication, a dashboard,
browsing and creating sales. The application provides a 
modern, responsive interface with a navigation bar, 
hero section on the Home page, and a footer on all pages.
Key pages include:
 Home, Dashboard, Browse Sales, Create Sale,
 About, and Contact.

\section{Part a: HTML and Django Template Code Implementation}
This section presents the HTML and Django template code used 
for the project layout.

\subsection{Base Template Explanation}
The base template, \texttt{base.html}, defines the overall structure of the site. 
It includes a navigation bar, a hero (jumbotron) section (on the Home page), and a footer. 
Content blocks are defined using Django's template block tags, allowing child templates to 
override specific sections.

\begin{verbatim}
<!-- templates/core/base.html -->
<!DOCTYPE html>
<html lang="en">
<head>
    <meta charset="UTF-8">
    <title>Django Sales App</title>
    <link rel="stylesheet" href="">
    <link rel="stylesheet" href="">
    
</head>
<body>
    <nav>...navbar code...</nav>
    
    <footer>...footer code...</footer>
    <script src=""></script>
</body>
</html>
\end{verbatim}

\subsection{Child Template Example}
A child template, such as \texttt{home.html}, extends the base template and fills in the 
content block:

\begin{verbatim}
<!-- templates/core/home.html -->

Home - Django Sales App

<div class="jumbotron">
    <h1>Welcome to the Django Sales App!</h1>
    <p>Browse and manage sales listings easily.</p>
</div>

\end{verbatim}

\subsection{Use of Template Tags}
Key Django template tags used include:
\begin{itemize}
    \item \verb||: Allows a template to inherit from a base template.
    \item \verb||: Defines sections that child templates can override.
    \item \verb||: Resolves the URL for static files (CSS, JS, images).
\end{itemize}
These tags help organize templates, promote code reuse, and simplify static file management.

\subsection{Bonus: URL and View Structure}
The \texttt{urls.py} and \texttt{views.py} files connect URLs to views and templates. 
For example:

\begin{verbatim}
# core/urls.py
from django.urls import path
from . import views

urlpatterns = [
    path('', views.home, name='home'),
    path('dashboard/', views.dashboard, name='dashboard'),
    path('sales/', views.browse_sales, name='browse_sales'),
    path('sales/create/', views.create_sale, name='create_sale'),
    path('about/', views.about, name='about'),
    path('contact/', views.contact, name='contact'),
]

# core/views.py
from django.shortcuts import render

def home(request):
    return render(request, 'core/home.html')
# ...other view functions...
\end{verbatim}

\section{Part b: Benefits of Django for Responsive Web Apps}
Django is ideal for building responsive web applications due to its template system, 
which separates logic from presentation. It integrates easily with CSS frameworks like 
Bootstrap, enabling mobile-friendly designs. Django also provides built-in security, 
scalability, and rapid development features, making it suitable for modern web apps.

\section{Part c: Linking Static CSS and JavaScript in Django}
To include static files in Django:
\begin{enumerate}
    \item Set \texttt{STATIC\_URL} in \texttt{settings.py} (e.g., \texttt{STATIC\_URL = '/static/'}).
    \item Add \verb|| at the top of each template.
    \item Reference static files using \verb|| or similar.
\end{enumerate}
Example:
\begin{verbatim}

<link rel="stylesheet" href="">
\end{verbatim}

\section{Part d: Components of MTV Architecture in Django}
Django follows the Model-Template-View (MTV) architecture:
\begin{itemize}
    \item \textbf{Model:} Defines data structure. E.g., the Sales model in \texttt{core/models.py} manages product data.
    \item \textbf{View:} Handles requests and returns responses. E.g., functions in \texttt{core/views.py} render templates for each page.
    \item \textbf{Template:} Renders HTML. E.g., files in \texttt{templates/core/} display content to users.
\end{itemize}

\end{document}
